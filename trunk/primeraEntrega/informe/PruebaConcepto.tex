\chapter{Prueba de Concepto}

\section{XXXX Datos de la prueba de concepto}

Completar con lo que fue la prueba de conecpto...
lo que son los rpc y el canal y bla bla bla
XXXXXXXXXXXXXXXXXXXXXXXXXXXXXXXXXX
XXXXXXXXXXXXXXXXXXXXXXXXXXXXXXXXXX
XXXXXXXXXXXXXXXXXXXXXXXXXXXXXXXXXX
XXXXXXXXXXXXXXXXXXXXXXXXXXXXXXXXXX
XXXXXXXXXXXXXXXXXXXXXXXXXXXXXXXXXX
XXXXXXXXXXXXXXXXXXXXXXXXXXXXXXXXXX
XXXXXXXXXXXXXXXXXXXXXXXXXXXXXXXXXX
XXXXXXXXXXXXXXXXXXXXXXXXXXXXXXXXXX
XXXXXXXXXXXXXXXXXXXXXXXXXXXXXXXXXX
XXXXXXXXXXXXXXXXXXXXXXXXXXXXXXXXXX
XXXXXXXXXXXXXXXXXXXXXXXXXXXXXXXXXX
XXXXXXXXXXXXXXXXXXXXXXXXXXXXXXXXXX
XXXXXXXXXXXXXXXXXXXXXXXXXXXXXXXXXX
XXXXXXXXXXXXXXXXXXXXXXXXXXXXXXXXXX
XXXXXXXXXXXXXXXXXXXXXXXXXXXXXXXXXX
XXXXXXXXXXXXXXXXXXXXXXXXXXXXXXXXXX
XXXXXXXXXXXXXXXXXXXXXXXXXXXXXXXXXX
XXXXXXXXXXXXXXXXXXXXXXXXXXXXXXXXXX
XXXXXXXXXXXXXXXXXXXXXXXXXXXXXXXXXX
XXXXXXXXXXXXXXXXXXXXXXXXXXXXXXXXXX
XXXXXXXXXXXXXXXXXXXXXXXXXXXXXXXXXX
XXXXXXXXXXXXXXXXXXXXXXXXXXXXXXXXXX
XXXXXXXXXXXXXXXXXXXXXXXXXXXXXXXXXX
XXXXXXXXXXXXXXXXXXXXXXXXXXXXXXXXXX
XXXXXXXXXXXXXXXXXXXXXXXXXXXXXXXXXX
XXXXXXXXXXXXXXXXXXXXXXXXXXXXXXXXXX
XXXXXXXXXXXXXXXXXXXXXXXXXXXXXXXXXX
XXXXXXXXXXXXXXXXXXXXXXXXXXXXXXXXXX
XXXXXXXXXXXXXXXXXXXXXXXXXXXXXXXXXX
XXXXXXXXXXXXXXXXXXXXXXXXXXXXXXXXXX
XXXXXXXXXXXXXXXXXXXXXXXXXXXXXXXXXX
XXXXXXXXXXXXXXXXXXXXXXXXXXXXXXXXXX
XXXXXXXXXXXXXXXXXXXXXXXXXXXXXXXXXX
XXXXXXXXXXXXXXXXXXXXXXXXXXXXXXXXXX
XXXXXXXXXXXXXXXXXXXXXXXXXXXXXXXXXX
XXXXXXXXXXXXXXXXXXXXXXXXXXXXXXXXXX
XXXXXXXXXXXXXXXXXXXXXXXXXXXXXXXXXX
XXXXXXXXXXXXXXXXXXXXXXXXXXXXXXXXXX
XXXXXXXXXXXXXXXXXXXXXXXXXXXXXXXXXX
XXXXXXXXXXXXXXXXXXXXXXXXXXXXXXXXXX
XXXXXXXXXXXXXXXXXXXXXXXXXXXXXXXXXX
XXXXXXXXXXXXXXXXXXXXXXXXXXXXXXXXXX
XXXXXXXXXXXXXXXXXXXXXXXXXXXXXXXXXX
XXXXXXXXXXXXXXXXXXXXXXXXXXXXXXXXXX
XXXXXXXXXXXXXXXXXXXXXXXXXXXXXXXXXX
XXXXXXXXXXXXXXXXXXXXXXXXXXXXXXXXXX
XXXXXXXXXXXXXXXXXXXXXXXXXXXXXXXXXX
XXXXXXXXXXXXXXXXXXXXXXXXXXXXXXXXXX
XXXXXXXXXXXXXXXXXXXXXXXXXXXXXXXXXX
XXXXXXXXXXXXXXXXXXXXXXXXXXXXXXXXXX
XXXXXXXXXXXXXXXXXXXXXXXXXXXXXXXXXX
XXXXXXXXXXXXXXXXXXXXXXXXXXXXXXXXXX
XXXXXXXXXXXXXXXXXXXXXXXXXXXXXXXXXX
XXXXXXXXXXXXXXXXXXXXXXXXXXXXXXXXXX
XXXXXXXXXXXXXXXXXXXXXXXXXXXXXXXXXX
XXXXXXXXXXXXXXXXXXXXXXXXXXXXXXXXXX
XXXXXXXXXXXXXXXXXXXXXXXXXXXXXXXXXX
XXXXXXXXXXXXXXXXXXXXXXXXXXXXXXXXXX
XXXXXXXXXXXXXXXXXXXXXXXXXXXXXXXXXX
XXXXXXXXXXXXXXXXXXXXXXXXXXXXXXXXXX
XXXXXXXXXXXXXXXXXXXXXXXXXXXXXXXXXX
XXXXXXXXXXXXXXXXXXXXXXXXXXXXXXXXXX
XXXXXXXXXXXXXXXXXXXXXXXXXXXXXXXXXX
XXXXXXXXXXXXXXXXXXXXXXXXXXXXXXXXXX
XXXXXXXXXXXXXXXXXXXXXXXXXXXXXXXXXX
XXXXXXXXXXXXXXXXXXXXXXXXXXXXXXXXXX
XXXXXXXXXXXXXXXXXXXXXXXXXXXXXXXXXX
XXXXXXXXXXXXXXXXXXXXXXXXXXXXXXXXXX
XXXXXXXXXXXXXXXXXXXXXXXXXXXXXXXXXX
XXXXXXXXXXXXXXXXXXXXXXXXXXXXXXXXXX
XXXXXXXXXXXXXXXXXXXXXXXXXXXXXXXXXX
XXXXXXXXXXXXXXXXXXXXXXXXXXXXXXXXXX
XXXXXXXXXXXXXXXXXXXXXXXXXXXXXXXXXX
XXXXXXXXXXXXXXXXXXXXXXXXXXXXXXXXXX
XXXXXXXXXXXXXXXXXXXXXXXXXXXXXXXXXX
XXXXXXXXXXXXXXXXXXXXXXXXXXXXXXXXXX
XXXXXXXXXXXXXXXXXXXXXXXXXXXXXXXXXX
XXXXXXXXXXXXXXXXXXXXXXXXXXXXXXXXXX
XXXXXXXXXXXXXXXXXXXXXXXXXXXXXXXXXX
XXXXXXXXXXXXXXXXXXXXXXXXXXXXXXXXXX
XXXXXXXXXXXXXXXXXXXXXXXXXXXXXXXXXX
XXXXXXXXXXXXXXXXXXXXXXXXXXXXXXXXXX
XXXXXXXXXXXXXXXXXXXXXXXXXXXXXXXXXX
XXXXXXXXXXXXXXXXXXXXXXXXXXXXXXXXXX
XXXXXXXXXXXXXXXXXXXXXXXXXXXXXXXXXX
XXXXXXXXXXXXXXXXXXXXXXXXXXXXXXXXXX
XXXXXXXXXXXXXXXXXXXXXXXXXXXXXXXXXX
XXXXXXXXXXXXXXXXXXXXXXXXXXXXXXXXXX
XXXXXXXXXXXXXXXXXXXXXXXXXXXXXXXXXX
XXXXXXXXXXXXXXXXXXXXXXXXXXXXXXXXXX
XXXXXXXXXXXXXXXXXXXXXXXXXXXXXXXXXX
XXXXXXXXXXXXXXXXXXXXXXXXXXXXXXXXXX
XXXXXXXXXXXXXXXXXXXXXXXXXXXXXXXXXX
XXXXXXXXXXXXXXXXXXXXXXXXXXXXXXXXXX
XXXXXXXXXXXXXXXXXXXXXXXXXXXXXXXXXX
XXXXXXXXXXXXXXXXXXXXXXXXXXXXXXXXXX
XXXXXXXXXXXXXXXXXXXXXXXXXXXXXXXXXX
XXXXXXXXXXXXXXXXXXXXXXXXXXXXXXXXXX
XXXXXXXXXXXXXXXXXXXXXXXXXXXXXXXXXX
XXXXXXXXXXXXXXXXXXXXXXXXXXXXXXXXXX

\section{Test de Integridad}

Para esta prueba de concepto se generaron tres test de integridad en forma de archivos batch.

Cada uno de estos test corresponde a un archivo .bat que va ejecutando las distintas parte de la comunicacion como la Recepci�nSegura, la TR, y el Canal.

Cada uno de ellos intenta abarcar uno de los casos posibles en la comunicaci�n entre las TR y la EC. 

\subsection{Test 01}

\begin{itemize}
	\item Descripcion: En este caso se levantan cinco TRs de las cuales todas funcionan bien menos una que luego de 45 segundos se cae y 50 segundos m�s tarde se vuelve a levantar.
	\item Parametros : Cada TR envia sus mensajes cada 20 segundos y la RecepcionSegura asume que si no recibe un mensaje luego de 40 segundos de una TR, la misma se cayo.
	\item Resultado : Este caso de test se corrio, y fue verificado que se detecto la caida y la vuelta en marcha de la TR. Asi tambien como se tomaron todos los mensajes correctamente.
\end{itemize}

\subsection{Test 02}

\begin{itemize}
	\item Descripcion: Este es un test que simula el comportamiento normal, tanto de las TR como de la EC. En este caso todas las TR permanecen levantas.
	\item Parametros : Cada TR envia sus mensajes cada 20 segundos y la RecepcionSegura asume que si no recibe un mensaje luego de 40 segundos de una TR, la misma se cayo.
	\item Resultado : Este caso de test se corrio y fue verificado que no se haya caido ninguna TR y que todos los mensajes llegaron correctamente.
\end{itemize}

\subsection{Test 03}

\begin{itemize}
	\item Descripcion: Este es un test que intenta simular la posibilidad de que el tiempo de llegada de los mensajes desde la TR hacia la EC sea muy alto y asi se asuma que la TR esta caida, pero luego de un tiempo recibe otro mensaje.
	\item Parametros : Cada TR envia sus mensajes cada 20 segundos y la RecepcionSegura asume que si no recibe un mensaje luego de 10 segundos de una TR, la misma se cayo. Se agrega a este caso de test tambien un delay entre que se levanta cada TR de 3 segundos.
	\item Resultado : Este caso de test se corrio y fue verificado que se asuma que la TR se cayo pero luego de recibir el siguiente mensaje se toma como reactiva nuevamente.
\end{itemize}

\section{Procesamiento y modelos}

En �sta secci�n vamos a mostrar una simulaci�n de lo que respecta a la parte de procesamiento de datos por los modelos matem�ticos.

La idea como se mencion� anteriormente en la arquitectura es poder incorporar distintos modelos matem�ticos sin necesidad de implementar cambios globales a nivel de sistema. Para ello lo que tratamos de simular fue un m�dulo que se encargara b�sicamente de trasformar los datos a procesar en estructuras que puedan ser utilizadas por cualquier modelo matem�tico respresentado por un conjunto de reglas.

En �ste caso se intent� armar un diccionario de modo tal que cada Terminal Remota( cuyo id corresponde a las claves del diccionario) provea un conjunto de mediciones (valores del diccionario) representada a su vez por otro diccionario en el cual las claves sean los datos sensados tales como temperatura, presi�n, etc y los valores sean las mediciones de dichos datos, es mdecir supungamos que tenemos una sola terminal remota que sensa presiones, entonces el diccionario representado viene a ser el siguiente:

Clave : 1 (Terminal remota cuyo id es 1), Valor : (Clave : presion, Valor : lista de valores sensados)

De �ste modo implentando �sta estructura de datos com�n, las reglas pertenecientes a un modelo matem�tico aplicar�n sobre �sta estructura de datos independientemente del modelo.

Para poder probar la implementaci�n lo que hacemos es correr los archivos de la siguiente manera:

\begin{verbatim}
python Modelos.py reglasX.py datosX.py(opcional)
\end{verbatim}

Lo que hace la sentencia anterior es cargar el m�dulo de procesamiento (Modelos.py), el conjunto de reglas a aplicar(reglas1.py, reglas2.py, etc) y finalmente los datos en forma de diccionario (Asumimos que ya vienen con la estructura que deseamos, que otro m�dulo (Parser) se encarga de llevar a cabo dicha tarea). 

Igualmente como se pude ver en la sentencia, la inclusi�n de datos es opcional, ya que de no proveer dicho archivo se generar� un diccionario con valores aleatorios de sensado.

Hicimos �stas dos posibles formas de obtener los datos para que sea sencillo el probado de las pruebas. Si generamos nuestros propios datos podremos analizar los diferentes resultados generados por las reglas, pero si por el contrario queremos valores aleatorios, tambi�n se brinda dicha posibilidad.

\subsection{Datos y casos de test}


Para probar las implementaciones, nos generamos b�sicamente dos archivos de datos: datos1.py y datos2.py y tambi�n dos conjuntos de reglas: reglas1.py( Que utiliza m�ximos para calcular) y reglas2.py( que utiliza promedios para calcular)
y lo que hicismos fue crear todas las posibles ejecuciones, es decir para cada regla cualquiera de los dos archivos de datos o ninguno. De �ste modo obtuvimos 6 casos de test que se pueden ejecutar directamente con los archivos Test X.bat que se encuentran en la carpeta casos de test y dentro de ella en el directorio Test Procesamiento.

Lo que se intent� llevar a cabo con �stas pruebas de test es el intercambio entre reglas de procesamiento de manera sencilla y no tanto as� el resultado de los procesamientos por lo cual no explayaremos los resultados obtenidos y daremos principal inter�s a la correcta modificabilidad de las reglas.