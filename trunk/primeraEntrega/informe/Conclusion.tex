\chapter{Conclusi�n}
En este Trabajo Pr�ctico, hicimos la prueba de hacer una parte del desarrollo de nuestro poryecto de software con RUP - Rational Unified Process. Fue un experimento interesante, ya que ninguno de nosotros habia estado involucrado en un proceso de desarrollo tan grande en nuestra vida profesional. La ventaja de utilizar un m�todo iterativo incremental es que nos permite hacer una iteraci�n, probando as� una parte completa del m�todo (y no solo haciendo una parte incompleta de �l).

En la pr�xima �tapa del TP, seguiremos con m�todos �giles, que a priori, nos da la sospecha de que nos van a gustar m�s. Veremos si esto se confirma o no.

\subsection{RUP: Fase de Elaboraci�n}
Con el fin de esta iteraci�n, cumplimos con algunos (la mayoria) de de los procesos previstos para la etapa de Elaboraci�n[
\begin{itemize}
	\item Analizamos el dominio del problema: tenemos la lista completa de los CU.
	\item Complementamos los requerimientos funcionales con la especificaci�n de Atributos de Calidad, mediante la t�cnica de escenarios.
	\item Establecimos una base para nuestra arquitectura: que documentamos mediante vistas y la probamos con un prototipo ejecutable.
	\item Desarollamos con cierto detalle el plan del proyecto.
	\item Armamos una lista de Riesgos del proyecto.
\end{itemize}

De todas formas, esto no significa que se haya finalizado la fase de Elaboraci�n, hay cosas que faltaron. Por ejemplo, habr�a que especificar en m�s detalle una porci�n importante de los Casos de Uso del Sistema para poder pasar a la siguiente fase, Construcci�n.