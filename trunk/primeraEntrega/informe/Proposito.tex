\chapter{Requerimientos funcionales y atributos de calidad}

\section{Prop�sito del sistema}

La idea del trabajo pr�ctico consiste b�sicamente en poder desarrollar un software consistente, que pueda satisfacer no s�lo las necesidades b�sicas de nuestro cliente plasmadas en requerimientos funcionales, sino que tambi�n pueda garantizar el cumplimiento de atributos de calidad que constituyen un factor fundamental en el correcto desempe�o de nuestro sistema al momento de salir a producci�n. 

El objetivo de la realizaci�n de dicho software es en principio, poder predecir, mediante la existencia de terminales remotas con sensores de medici�n y modelos matem�ticos de procesamiento, el estado del tiempo en la totalidad del pa�s, poniendo principal �nfasis en generar alertas en caso de tormentas el�ctricas y huracanes a tiempo para poder evacuar las zonas afectadas por dichos fen�menos. Sin embargo es muy fuerte la correspondencia entre las funcionalidades a desarrollar para satisfacer dicho objetivo y atributos de calidad cr�ticos en este contexto. Por ejemplo la disponibilidad es un factor fundamental para el cumplimiento del prop�sito.

En esta parte del informe intentaremos destacar, aquellas funcionalidades necesarias en el sistema, as� como los atributos de calidad instanciados en escenarios que se pudieron obtener del relevamiento del contexto.


\newpage
\input{CasosDeUso}
\clearpage

\newpage
\chapter{Escenarios}

En esta secci�n se mostrar�n y detallar�n los diferentes escenarios para los atributos de calidad identificados en el sistema a desarrollar. Estos, consideramos que son los principales, y que definir�n la arquitectura. Mas adelante se mostrar� el diagrama de arquitectura pensada para este sistema y las t�cticas usadas para cubrir los principales atriburos de calidad encontrados.

Se h�n indentificado 6 atributos de calidad en total, estos son: performace, disponibilidad, confiabilidad, modificabilidad, interoperabilidad, usabilidad. Cada uno tiene 1 o mas escenarios en los cuales cada atributo aplica.

\begin{itemize}
\item Atributos de Performace:

    \begin{enumerate}
    \item La informaci�n critica de los resultados del modelo matem�tico se en env�a lo mas rapido posible a la pagina del ministerio.\\ \\
    \textbf{Fuente:} interna (alarma).\\    
    \textbf{Estimulo:} El modelo matem�tico genera una predicci�n de los fenomenos meteorologicos critica.\\    
    \textbf{Artefacto:} Procesamiento (Estaci�n central), Datos(sistema de monitoreo).\\    
    \textbf{Entorno:} Normal/Sobrecarga.\\    
    \textbf{Respuesta:} Se envia la informaci�n critica al ministerio.\\    
    \textbf{Medici�n de la respuesta:} La informaci�n relacionada con la predicci�n se envia en a lo sumo 30 segundos.
    
    \item El procesamiento de los datos obtenidos para generar una predicci�n se realiza lo mas rapido posible.\\ \\
    \textbf{Fuente:} interna (alarma).\\
    \textbf{Estimulo:} Pedido de procesamiento de los datos almacenados.\\
    \textbf{Artefacto:} Procesamiento (Estaci�n central).\\
    \textbf{Entorno:} Sobrecarga.\\
    \textbf{Respuesta:} Los datos son procesados y se devuelve un resultado.\\
    \textbf{Medici�n de la respuesta:} El resultado se procesa en menos de 1 minuto.
    \end{enumerate}
    
\item Atributos de disponibilidad:

    \begin{enumerate}
    \item Los mensajes que se reciben en forma correcta son procesados.\\ \\
    \textbf{Fuente:} TR, Biggest Satelite, Sistema eolico.\\
    \textbf{Estimulo:} Se recibe un mensaje con datos capturados.\\
    \textbf{Artefacto:} Data manager (Estaci�n central).\\
    \textbf{Entorno:} Normal.\\
    \textbf{Respuesta:} El mensaje es procesado.\\
    \textbf{Medici�n de la respuesta:} El mensaje es procesado y almacenado en el 99,99\% de las veces.
    
    \item La informaci�n del sistema esta disponible para los clientes externos o usu�rios.\\ \\
    \textbf{Fuente:} Cliente externo, usu�rio.\\
    \textbf{Estimulo:} Pedido de datos almacenados.\\
    \textbf{Artefacto:} Datos (Sistema de monitoreo).\\
    \textbf{Entorno:} Normal.\\
    \textbf{Respuesta:} Los datos son enviados a los que lo solicitaron.\\
    \textbf{Medici�n de la respuesta:} Los datos se encuentran disponibles en el 99,999\% de los casos.
    
    \item Los resultados de los modelos son cosistentes .\\ \\
    \textbf{Fuente:} Componente de procesamiento.\\
    \textbf{Estimulo:} Resultado de modulo invalido.\\
    \textbf{Artefacto:} Sistema de corroborador de resultados(Estaci�n central).\\
    \textbf{Entorno:} Normal.\\
    \textbf{Respuesta:} Se detectan los errores y no se modifica el estado actual consistente.\\
    \textbf{Medici�n de la respuesta:} El sistema tiene resultados consistentes en el 99,999\% de los casos.
    
    \item Los mensajes desordenados son ordenados correctamente.\\ \\
    \textbf{Fuente:} Terminal Remota.\\
    \textbf{Estimulo:} Llegan datos desordenados.\\
    \textbf{Artefacto:} Recepcion segura (Estaci�n central).\\
    \textbf{Entorno:} Normal.\\
    \textbf{Respuesta:} Los mesajes son ordenados correctamente.\\
    \textbf{Medici�n de la respuesta:} Los mensajes desordenados son ordenados correctamente en el 99,999\% de los casos.    
    \end{enumerate}
    
\item Atributos de confiabilidad:

    \begin{enumerate}
    \item Los resultados obtenidos son correctos.\\ \\
    \textbf{Fuente:} Componente interna.\\
    \textbf{Estimulo:} Se realiza un pedido de la informaci�n meteorologica.\\
    \textbf{Artefacto:} Procesamiento (Estaci�n central).\\
    \textbf{Entorno:} Normal.\\
    \textbf{Respuesta:} Los datos almacenados son procesados y se devuelve un resultado(pronostico).\\
    \textbf{Medici�n de la respuesta:} El pronostico se confirma en un 99,9\% de los casos.
    \end{enumerate}

\item Atributos de seguridad:

    \begin{enumerate}
    \item Los mensajes que arriban a la Estaci�n central corruptos o erroneos son descartados.\\ \\
    \textbf{Fuente:} Intruso, usu�rio, TR.\\
    \textbf{Estimulo:} Arriba un mensaje alterado, corrupto o erroneo.\\
    \textbf{Artefacto:} Recepci�n segura (Estaci�n central).\\
    \textbf{Entorno:} Online.\\
    \textbf{Respuesta:} Se detecta el mensaje alterado, corrupto o erroneo y se descarta.\\
    \textbf{Medici�n de la respuesta:} Los mensajes alterados, corruptos o erroneos, son descartados en el 99,99\% de los casos.
    \end{enumerate}

\item Atributos de modificabilidad:

    \begin{enumerate}
    \item La modificaci�n del modelo se puede realizar en un lapso de tiempo aceptable.\\ \\
    \textbf{Fuente:} Usuario, experto, desarrollador.\\
    \textbf{Estimulo:} Se quiere modificar el modelo matem�tico.\\
    \textbf{Artefacto:} Procesamiento (Estaci�n central).\\
    \textbf{Entorno:} Ejecuci�n\\
    \textbf{Respuesta:} El modelo matem�tico es modificado.\\
    \textbf{Medici�n de la respuesta:} La modificaci�n se realiza en a lo sumo 72 horas.

		\item Se pueden instalar TRs o sensores en menos de 10 d�as.\\ \\
    \textbf{Fuente:} Empresa.\\
    \textbf{Estimulo:} Instalaci�n de nuevas TRs o sensores.\\
    \textbf{Artefacto:} Controlador Estado (Estaci�n central).\\
    \textbf{Entorno:} Normal.\\
    \textbf{Respuesta:} Se instalan las componentes solicitadas.\\
    \textbf{Medici�n de la respuesta:} La instalaci�n se realiza en menos de 10 d�as.

    \item La agenda de una TR se realiza correctamente.\\ \\
    \textbf{Fuente:} Usuario u Operario.\\
    \textbf{Estimulo:} Cambio de la agenda de un sensor.\\
    \textbf{Artefacto:} Configuraci�n agenda (Terminal Remota).\\
    \textbf{Entorno:} Ejecuci�n.\\
    \textbf{Respuesta:} La agenda se configura adecuadamente.\\
    \textbf{Medici�n de la respuesta:} La configuraci�n se realiza en menos de 60 segundos.

    \end{enumerate}
    
\item Atributos de usabilidad:

   \begin{enumerate}
   \item La informaci�n del sistema brinda al los usuarios.\\ \\
   \textbf{Fuente:} Usuario final / empresa.\\
   \textbf{Estimulo:} Acceso a la informaci�n mediante Google Maps, estado del tiempo, etc.\\
   \textbf{Artefacto:} Servicios p�gina (Sistema de monitoreo).\\
   \textbf{Entorno:} Ejecuci�n.\\
   \textbf{Respuesta:} Se brinda el servicio solicitado.\\
   \textbf{Medici�n de la respuesta:} El usuario obtiene un 90\% de satisfacci�n.
   
   \item El acceso a la informaci�n del sistema se hace facilmente.\\ \\
   \textbf{Fuente:} Empleado.\\
   \textbf{Estimulo:} Acceso a la informaci�n de estadofacilmente.\\
   \textbf{Artefacto:} Visualizaci�n (Sistema de monitoreo).\\
   \textbf{Entorno:} Ejecuci�n.\\
   \textbf{Respuesta:} La informaci�n es brindada correctamente.\\
   \textbf{Medici�n de la respuesta:} El usuario obtiene un 90\% de satisfacci�n.
  
   \end{enumerate}
   
\item Atributos de interoperabilidad:

   \begin{enumerate}
   \item El acceso a la informaci�n del sistema se hace facilmente.\\ \\
   \textbf{Fuente:} Sistema eolico.\\
   \textbf{Estimulo:} Llegan datos al sistema.\\
   \textbf{Artefacto:} Arbitro 1 (Estaci�n central).\\
   \textbf{Entorno:} Normal.\\
   \textbf{Respuesta:} Se procesan los datos y se seleccionan los mas precisos.\\
   \textbf{Medici�n de la respuesta:} Los datos se reciben en el 99,99\% de los casos.
  
   \end{enumerate}
   
\end{itemize}
\clearpage