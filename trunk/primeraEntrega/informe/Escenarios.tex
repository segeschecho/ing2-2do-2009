\chapter{Escenarios}

En esta secci�n se mostrar�n y detallar�n los diferentes escenarios para los atributos de calidad identificados en el sistema a desarrollar. Estos, consideramos que son los principales, y que definir�n la arquitectura. Mas adelante se mostrar� el diagrama de arquitectura pensada para este sistema y las t�cticas usadas para cubrir los principales atriburos de calidad encontrados.

Se h�n indentificado 4 atributos de calidad en total, estos son: performace, disponibilidad, confiabilidad y modificabilidad. Cada uno tiene 1 o mas escenarios en los cuales cada atributo aplica.

\begin{itemize}
\item Atributos de Performace:

    \begin{enumerate}
    \item La informaci�n critica de los resultados del modelo matem�tico se en env�a lo mas rapido posible a la pagina del ministerio.\\ \\
    \textbf{Fuente:} interna (alarma).\\    
    \textbf{Estimulo:} El modelo matem�tico genera una predicci�n de los fenomenos meteorologicos critica.\\    
    \textbf{Artefacto:} Sistema de monitoreo.\\    
    \textbf{Entorno:} Normal/Sobrecarga.\\    
    \textbf{Respuesta:} Se envia la informaci�n critica al ministerio.\\    
    \textbf{Medici�n de la respuesta:} La informaci�n relacionada con la predicci�n se envia en a lo sumo 30 segundos.
    
    \item El procesamiento de los datos obtenidos para generar una predicci�n se realiza lo mas rapido posible.\\ \\
    \textbf{Fuente:} interna (alarma).\\
    \textbf{Estimulo:} Pedido de procesamiento de los datos almacenados.\\
    \textbf{Artefacto:} Estaci�n central (procesamiento).\\
    \textbf{Entorno:} Sobrecarga.\\
    \textbf{Respuesta:} Los datos son procesados y se devuelve un resultado.\\
    \textbf{Medici�n de la respuesta:} El resultado se procesa en menos de 1 minuto.
    \end{enumerate}
    
\item Atributos de disponibilidad:

    \begin{enumerate}
    \item Los mensajes que se reciben en forma correcta son procesados.\\ \\
    \textbf{Fuente:} TR, Biggest Satelite, Sistema eolico.\\
    \textbf{Estimulo:} Se recibe un mensaje con datos capturados.\\
    \textbf{Artefacto:} Estaci�n central.\\
    \textbf{Entorno:} Normal.\\
    \textbf{Respuesta:} El mensaje es procesado.\\
    \textbf{Medici�n de la respuesta:} El mensaje es procesado y almacenado en el 99,99\% de las veces.
    
    \item La informaci�n del sistema esta disponible para los clientes externos o usu�rios.\\ \\
    \textbf{Fuente:} Cliente externo, usu�rio.\\
    \textbf{Estimulo:} Pedido de datos almacenados.\\
    \textbf{Artefacto:} Sistema de monitoreo.\\
    \textbf{Entorno:} Normal.\\
    \textbf{Respuesta:} Los datos son enviados a los que lo solicitaron.\\
    \textbf{Medici�n de la respuesta:} Los datos se encuentran disponibles en el 99,999\% de los casos.
    \end{enumerate}
    
\item Atributos de confiabilidad:

    \begin{enumerate}
    \item Los resultados obtenidos son correctos.\\ \\
    \textbf{Fuente:} Componente interna.\\
    \textbf{Estimulo:} Se realiza un pedido de la informaci�n meteorologica.\\
    \textbf{Artefacto:} Estaci�n central.\\
    \textbf{Entorno:} Normal.\\
    \textbf{Respuesta:} Los datos almacenados son procesados y se devuelve un resultado(pronostico).\\
    \textbf{Medici�n de la respuesta:} El pronostico se confirma en un 99,9\% de los casos.
    \end{enumerate}

\item Atributos de seguridad:

    \begin{enumerate}
    \item Los mensajes que arriban a la Estaci�n central corruptos o erroneos son descartados.\\ \\
    \textbf{Fuente:} Intruso, usu�rio, TR.\\
    \textbf{Estimulo:} Arriba un mensaje alterado, corrupto o erroneo.\\
    \textbf{Artefacto:} Estaci�n central.\\
    \textbf{Entorno:} Online.\\
    \textbf{Respuesta:} Se detecta el mensaje alterado, corrupto o erroneo y se descarta.\\
    \textbf{Medici�n de la respuesta:} Los mensajes alterados, corruptos o erroneos, son descartados en el 99,99\% de los casos.
    \end{enumerate}

\item Atrbutos de modificabilidad:

    \begin{enumerate}
    \item La modificaci�n del modelo se puede realizar en un lapso de tiempo aceptable.\\ \\
    \textbf{Fuente:} Usuario, experto, desarrollador.\\
    \textbf{Estimulo:} Se quiere modificar el modelo matem�tico.\\
    \textbf{Artefacto:} Estaci�n central (computo).\\
    \textbf{Entorno:} \textcolor{red}{\large \textbf{Normal??}}\\
    \textbf{Respuesta:} El modelo matem�tico es modificado.\\
    \textbf{Medici�n de la respuesta:} La modificaci�n se realiza en a lo sumo 72 horas.
    \end{enumerate}

\end{itemize}