\chapter{Arquitectura}

En esta parte se desarrollar� la arquitectura general del producto. Analiceremos estilos arquitect�nicos, las t�cticas correspondientes para poder contemplar los atributos de calidad y finalmente diferentes estructuras que nos conllevar�n a realizar las vistas arquitect�nicas apropiadas para lograr un buen entendimiento de las diferentes partes del sistema.

\section{Componentes y conectores}

La primera de las estructuras que vamos a analizar es la asociada a entidades run-time. Para ello utilizaremos vistas de componentes y conectores que nos permitan identificar �stas entidades y poder instanciarlas como componentes o conectores. Esto nos dar� una idea de como los diferentes componentes contribuyen a la ejecuci�n de las funcionalidades solicitadas y como las t�cticas aplicadas se corresponden al control de los diferentes atributos de calidad. Presentaremos tres vistas de acuerdo a los componentes que funcionan en cada uno de las tres grandes partes que componen nuestro sistema(Terminal Remota, Estaci�n central y Sistema de Monitoreo) y tambi�n explicaremos la forma de comunicaci�n y flujo de datos entre dichas partes. Cada una de las tres vistas se encuentran al final de �sta secci�n por lo que es aconsejable analizarla en paralelo a medida que se lee la explicaci�n del funcionamiento de las componentes internas.

\subsection{Funcionamiento de la Estaci�n Central}


\subsubsection{T�cticas elegidas en la Estaci�n Central}


\subsection{Funcionamiento del Sistema de Monitoreo}

El Sistema de Monitoreo es la parte del sistema encargada de la interfaz con los usuarios, clientes y otros sistemas externos. En este sector se prestar�n diferentes servicios, ya sea mediante una interfaz de comunicaci�n directa o mediante Web Services. Estos servicios tienen que ver, en su mayor�a, a la visualizaci�n de los datos que el sistema tenga disponible en el momento. �stos datos pueden ser:

\begin{itemize}
\item Los resultados de los m�todos procesados por la Estaci�n Central, que pueden estar destinados a clientes externos, como por ejemplo la p�gina del Ministerio de Infraestructura, o para otros clientes externos que se espera tener en un futuro, como es el caso de la empresa AgroTop.

\item Los datos que se refieren al estado general del sistema en tiempo real, el estado de la conexi�n entre las TRs y la Estaci�n central, el estado individual de cada TR, cuales de �stas est�n funcionando correctamente, cuales fueron trianguladas, y cuales no fueron trianguladas y se esta utilizando el sistema satelital Biggest Satelite para obtener los datos de su zona.

\item La informaci�n que es enviada en forma de mail a los responsables por alertas que se producen.

\item La recepci�n de datos provenientes del Sistema E�lico, que son utilizados por la Estaci�n Central para discernir que datos son mas precisos.

\end{itemize}

Como se puede observar, el Sistema de Monitoreo es la gran interfaz que tiene el sistema.

\subsubsection{Descripci�n de los componentes del Sistema de Monitoreo}

En �sta secci�n se nombrar�n y explicar�n los componentes que forman parte del Sistema de Monitoreo, asi tambi�n como su rol dentro de la arquitectura.

\begin{itemize}

\item \textbf{Interfaz SE:} Este componente es el encargado de 	comunicarse con el Sistema E�lico, implementando su protocolo de comunicaci�n, y a su vez se encarga de comunicarse con la Estaci�n Central, para enviarle los datos recibidos al componente encargado del procesamiento de los mismos.

\end{itemize}

\subsubsection{T�cticas elegidas en el Sistema de Monitoreo}


\subsection{Funcionamiento de las Terminales Remotas}


\subsubsection{T�cticas elegidas en las Terminales Remotas}


\section{Comunicaci�n entre la Estaci�n Central y el Sistema de Monitoreo}