\documentclass{beamer}

\usetheme{Warsaw}
\usepackage[spanish,activeacute]{babel}
\usepackage[latin1]{inputenc}

\title[No silver bullet]{No Silver Bullet. Essence and Accidents in Software Engineering.}
\author[Castillo, Elizalde, Gonzalez, Page]{Gonzalo Castillo, Victoria Elizalde, Segio Gonzalez, Mart��n Page}
\date[10 de Noviembre 2009 ]{Fred Brooks es Cient��fico de la Computaci�n e Ingeniero de Sofware. Recibi� el Turing Award en 1999.}
\institute{FCEyN}

\begin{document}

\frame{\titlepage}

\section{}
\subsection{}
\frame
{
  \frametitle{Introducci�n}
  \begin{itemize}
  \item <1->\textcolor{blue}{El paper} No Silver Bullet. Essence and Accidents in Software Engineering. IEEE Computer, Abril de 1987.
  \item <2->\textcolor{blue}{Quienes Somos?} Gonzalo Castillo, Victoria Elizalde, Sergio Gonzalez y Mart�n Page
  \item <3->\textcolor{blue}{Y Brooks?} Fred Brooks es Cient��fico de la Computaci�n e Ingeniero de Sofware. Recibi� el Turing Award en 1999 y es conocido por haber escrito el libro The Mythical Man-Month, adem�s de No silver bullet.
  \end{itemize}
}
\frame
{
  \frametitle{La met�fora}
  \begin{itemize}
  \item<1-> Brooks compara un proyecto de software con un hombre lobo: algo inocente se transforma en un monstruo.
  \item<2-> Necesidad de una "bala de plata", algo que haga bajar costos y aumente productividad, confiabilidad y simplicidad.
  \item<3-> Mayor dificultad del sofware: la especificaci�n, dise�o y testing de la estructura conceptual.
  \end{itemize}
}
\frame
{
  \frametitle{Las dificultades}
  \begin{itemize}
  \item<2-> \emph{Dificultades Esenciales}: Inherentes a la naturaleza misma del software.
  	\begin{itemize}
  \item<2-> \textbf{Complejidad}: Intr�nseca del software(escalabilidad, numeraci�n de estados, comunicaci�n).
  \item<3-> \textbf{Conformidad}: El software debe cumplir con limitaciones arbitrarias impuestas por personas y reglas de negocio.
  \item<4-> \textbf{Modificabilidad}: El software siempre va a estar sujeto a cambios.    
  \item<5-> \textbf{Invisibilidad}:El software es invisible e individualizable en el espacio. El software se intuye, pero no se ve.
  \end{itemize}
  \item<6-> \emph{Dificultades accidentales}: Dificultades no inherentes al sotware sino a su producci�n(Ej.Tipo de lenguaje de programaci�n).
  \end{itemize}
}

\frame
{
  \frametitle{Avances que resolvieron dificultades accidentales}
  \begin{itemize}
  \item<1-> \textcolor{red}{Lenguajes de Alto Nivel}: Abstracciones conceptuales. Esconden complejidad accidental del programa compilado.
  \item<2-> \textcolor{red}{Time-Sharing}: La posibilidad de compartir el tiempo de ejecuci�n entre procesos combate el accidente de
  los programas batch.
  \item<3-> \textcolor{red}{Ambientes de desarrollo unificado}: Combaten el accidente de tener aplicaciones que resuelven en forma
  individual las problem�ticas comunes (bibliotecas integradas, formatos de archivos unificados, tuber�as y filtros).
  \end{itemize}
}

\frame
{
  \frametitle{Esperanzas y potenciales balas de plata}
  \begin{itemize}
  \item<1-> \textbf{Lenguajes de alto nivel y POO}: LA jerarquizaci�n y abstracci�n(TADs) ataca tener enormes expresiones sint�cticas.
  \item<2-> \textbf{Inteligencia Arificial y Sistemas Expertos}: Conjunto de reglas de base y motor de inferencia para facilitar el desarrollo a principiantes.
  \item<3-> \textbf{Programaci�n Autom�tica}: A partir de especificaciones generar c�digo. Inviable y poco generalizable.
  \item<4-> \textbf{Programaci�n Visual}: Inviable por la invisibilidad del software.
  \item<5-> \textbf{Verificaci�n de Programas}: Antes de la etapa de contrucci�n, lo cual reduce el proceso de testing. Muy importante la validaci�n.
  \item<6-> \textbf{Entornos y herramientas}: Facilitan el trabajo de los desarrolladores(Ej. reducen errores sint�cticos).
  \end{itemize}
}

\frame
{
  \frametitle{Ataques a dificultades escenciales}
  \begin{itemize}
  \item<+-| alert@+>\textbf{Comprar vs Contruir}: Si la construcci�n de software es tan dif�cil, entonces compremoslo!!.
  \item<+-| alert@+> \textbf{Prototipos y refinaci�n de requerimientos}: El cliente no sabe lo que quiere. Es muy importante el feedback.
  \item<+-| alert@+> \textbf{Desarrollo iterativo incremental}: Es imposible contruir el producto en su totalidad de manera inmediata.
  \item<+-| alert@+>\textbf{Buenos dise�adores}: Buen dise�ador = buen dise�o. Fomentar el crecimiento de buenos dise�adores.
  \end{itemize}
}

\frame
{
  \frametitle{Conclusiones}
  Brooks fue un adelantado a su �poca, hace 20 a�os atr�s tuvo visiones que incluso hoy en d�a tienen gran relevancia en el desarrollo del software.
  \begin{itemize}
  \item<1-> \emph{Validaci�n usando prototipos.}
  \item<2-> \emph{Procesos iterativos incrementales(PU, Scrum).}
  \item<3-> \emph{Importancia de los buenos dise�os.}
  \end{itemize}
  <4->  Reflexi�n del grupo:
  <4->  Comprar vs Construir �El futuro?
  
}
\end{document}
