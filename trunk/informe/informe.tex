 %%	SECCION documentclass																									 %%	
%%---------------------------------------------------------------------------%%
\documentclass[a4paper]{report}

%%---------------------------------------------------------------------------%%
%%	SECCION usepackage																											 %%	
%%---------------------------------------------------------------------------%%
\usepackage{amsmath, amsthm}
\usepackage[spanish,activeacute]{babel}
\usepackage{caratula}
\usepackage{a4wide}
\usepackage{hyperref}
\usepackage{fancyhdr}
\usepackage{graphicx} % Para el logo magico!
\usepackage{amssymb}
\usepackage{amsmath}
\usepackage[latin1]{inputenc}
\usepackage[dvipsnames,usenames]{color}
\usepackage{amsfonts}
\usepackage{ulem}
%\usepackage{highlight}
\usepackage{fancybox}
%\usepackage{marvosym}

%%---------------------------------------------------------------------------%%
%%	SECCION opciones																												 %%	
%%---------------------------------------------------------------------------%%
\parskip    = 11 pt
\headheight	= 13.1pt
\pagestyle	{fancy}
\definecolor{orange}{rgb}{1,0.5,0}

\addtolength{\headwidth}{1.0in}

\addtolength{\oddsidemargin}{-0.5in}
\addtolength{\textwidth}{1.0in}
\addtolength{\topmargin}{-0.5in}
\addtolength{\textheight}{0.7in}

%%---------------------------------------------------------------------------%%
%%	SECCION document	 %%	
%%---------------------------------------------------------------------------%%
\begin{document}
\renewcommand{\chaptername}{Parte }

%%---- Caratula -------------------------------------------------------------%%
\materia{Ingenier�a del Software II (2do cuatrimestre de 2009)}
\titulo{Trabajo Pr�ctico I - Preentrega}

\integrante{Castillo, Gonzalo}{164/06}{gonzalocastillo\_086@hotmail.com}
\integrante{Elizalde Victoria}{}{}
\integrante{Gonzalez Sergio}{}{}
\integrante{Page Saal Mart�n}{}{}
\resumen{
Completar resumen}

% TOC, usa estilos locos
\maketitle
\pagestyle{empty}
{
\fancypagestyle{plain}
    {
    \fancyhead{}
    \fancyfoot{}
    \renewcommand{\headrulewidth}{0.0pt}
    } % clear header and footer of plain page because of ToC
\tableofcontents
}

\newpage
% arreglos los estilos para el resto del documento, y
% reseteo los numeros de pagina para que queden bien
\pagenumbering{arabic}
\fancypagestyle{plain} {
    \fancyhead[LO]{Castillo, Elizalde, Gonzalez, Page Saal}
    \fancyhead[C]{}
    \fancyhead[RO]{P\'agina \thepage\ de \pageref{LastPage}}
    \fancyfoot{}
    \renewcommand{\headrulewidth}{0.4pt}
}
\pagestyle{plain}

\newpage
\chapter{Descripci'on de Casos de Uso:}

En el siguiente texto se mostrar'an los casos de uso principales identificados en el sistema, junto con su descripci'on. En algunos casos, se incluir'an tambien los diferentes actores, que participan en el mismo, 'estos se corresponden con los actores que se muestran en el diagrama de contexto del sistema.


\section{Casos de uso dentro de las TRs:}

\begin{enumerate}
\item Configurando agenda en sensor:
	\begin{itemize}
	\item Agentes: Estaci'on Central, Sensor.
	\item Descripci'on: La estaci�n central indica que se tiene que modificar la agenda de uno de los sensores de la TR, esta recibe las instrucciones, y modifica dicha agenda utilizando el protocolo de conexi�n con el sensor.
	\end{itemize}
	
\item Obteniendo datos censados:
	\begin{itemize}
	\item Agentes: Sensor.
	\item Descripci'on: La TR recibe un dato de un sensor y lo almacena para luego cuando llegue el tiempo correcto lo envie a la estacion central.
	\end{itemize}
	
\item Monitoreando/configurando consumo de energ�a:
	\begin{itemize}
	\item Agentes: Componente muy usada del mercado, Bater�a, Panel solar)
	\item Descripci'on: {\bf COMPLETAR}
	\end{itemize}

\item Detectando ca�da del sensor:
	\begin{itemize}
	\item Agentes: Estaci'on Central, Sensor.
	\item Descripci'on: Se detecta la caida de un sensor cuando la TR no recibe datos del mismo dentro del tiempo que tiene configurado para recibirlos. {\bf COMPLETAR}
	\end{itemize}
	
\item Detectando recuperaci�n del sensor:
	\begin{itemize}
	\item Agentes: Estaci'on Central, Sensor.
	\item Descripci'on:	Se detecta la reactivaci�n de un sensor cuando se comienza a recibir informaci�n del mismo. {\bf COMPLETAR}
	\end{itemize}
	
\item ABM de sensor:
	\begin{itemize}
	\item Agentes: Sensor, Operario, {\bf Estaci'on Central?}
	\item Descripci'on: Se instala un nuevo sensor en la TR y el operario la configura para que en determinado momento se comience a enviar sus datos a la Estaci�n Central.
	\end{itemize}

\item Detecci�n de posible ca�da por falta de energ�a:
	\begin{itemize}
	\item Agentes: Componente muy usada del mercado, Estaci'on Central.
	\item Descripci'on: El componente muy usado del mercado le indica a la TR que la energ�a en la bater�a se encuentra en un nivel bajo y es necesario recargarla, en este caso ls TR indica esto a la estaci�n central. {\bf COMPLETAR}
	\end{itemize}

\item Detectando ca�da del sensor
	\begin{itemize}
	\item Agentes:
	\item Descripci'on:
	\end{itemize}
	
\end{enumerate}

\label{LastPage}
\end{document}
