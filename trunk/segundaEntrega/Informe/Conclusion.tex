\chapter{Conclusi�n}
Como conclusiones generales del trabajo podemos rescatar algunas cosas. Con respecto a la utilizaci�n del m�todo �gil scrum, nos pareci� un buen m�todo que permite mejorar la productividad. El hecho de que fomente la auto-organizaci�n, hace que se distribuyan naturalmente las tareas entre el equipo. Esto lo consideramos muy importante, ya que se pueden aprovechar mucho mas los recursos que se tienen.  La forma de seguimiento que usa Scrum nos pareci� muy buena y totalmente novedosa: se va viendo d�a a d�a cuantas horas faltan para completar cada tarea en vez de cuantas horas se trabaj�. Las stand up meetings nos parecen un concepto interesante. 

De todas formas, cuando hicimos una reuni�n con nuestro corrector termin� siendo mas una reuni�n de avance que de otra cosa, y en general cuando nos reun�amos nosotros tambi�n nos pasaba que la reuni�n siempre terminaba siendo m�s larga de lo esperado (no solo habl�bamos de qu� hac�amos y de los obst�culos que ten�amos, sino tambi�n de c�mo y hasta a veces nos pon�amos a implementar algo entre todos). 

Lo que no nos gust� tanto de este TP es que el proceso no fue tan real: por ejemplo, los casos de aceptaci�n y las user-stories las tuvimos que definir nosotros, no nos las dio el cliente. Igual, entendemos que siendo un trabajo pr�ctico no se pod�a implementar SCRUM tal cual sino que se ten�an que hacer ciertas adaptaciones. Pero, en el trabajo anterior hubo por ejemplo un QAW, y si bien no nos entregaron escenarios especificados en detalle, si nos dieron una lista de atributos de calidad priorizados.  Tampoco definimos qu� user stories iban en el sprint, pero, de nuevo, trat�ndose de un trabajo pr�ctico, nos parece razonable.  