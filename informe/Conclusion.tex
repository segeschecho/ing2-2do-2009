\chapter{Conclusi�n}

Si bien se sabe que el dise�o del proyecto est� en sus primeros pasos, con lo realizado en esta preentrega, se tiene una especificaci�n relativamente exhaustiva de las cosas mas importantes del dominio del problema. Con estos datos se pudo definir la base de la arquitectura del sistema, que si bien puede cambiar, se supone que no afectara gravemente el dise�o original, ni traer� problemas que posterguen demasiado el proyecto.

Para realizar esta afirmaci�n nos basamos en que al realizar el diagrama contextual (que sirve en este caso de arquitectura), tomamos en cuenta varios de los requerimientos y atributos de calidad planteados inicialmente, proponiendo una arquitectura que es incremental, en cuanto a que se pueden agregar Terminales Remotas, y en cuanto a que se puede incrementar el poder de c�mputo. Tambi�n es resistente a fallas en las transmisiones, ya que se implementara un protocolo confiable entre las TRs y la Estaci�n Central y tambi�n es resistente a fallas en los equipos, dado que se van a poder subsanar la ca�da de una TR triangulando o utilizando un servicio externo. Adem�s se contempla el hecho de tener comunicaci�n con diferente tipo de sistemas, ya sean clientes externos (como AgroTop), proveedores de servicio (Biggest Satelite) y usuarios internos, encargados de monitorear el estado del sistema.

Se dise�� un plan que tiene en cuenta tiempos de entrega y de desarrollo que son acotados, por esta raz�n se tuvo que identificar dependencias entre las funcionalidades principales que se detectaron, como as� su importancia y complejidad dentro del sistema.

Como se conoce estos m�todos llevan bastante tiempo, pero ayudan a tener un control del proyecto y tener luego un hilo de ejecuci�n bien definido que permitir� optimizar el desarrollo del sistema durante el tiempo. Teniendo en cuenta el contexto en el que se realiz� el trabajo creemos haber cumplido las expectativas en cuanto a los requerimientos pedidos para esta entrega.
