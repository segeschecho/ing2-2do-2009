\chapter{Escenarios}

En esta secci�n se mostrar�n y detallar�n los diferentes escenarios para los atributos de calidad identificados en el sistema a desarrollar.

\begin{enumerate}
\item Atributos de Performace:

    \begin{itemize}
    \item La informaci�n critica de los resultados del modelo matem�tico se en env�a lo mas rapido posible a la pagina del ministerio.\\    
    \textbf{Fuente:} interna (alarma).\\    
    \textbf{Estimulo:} El modelo matem�tico genera una predicci�n de los fenomenos meteorologicos critica.\\    
    \textbf{Artefacto:} Sistema de monitoreo.\\    
    \textbf{Entorno:} Normal/Sobrecarga.\\    
    \textbf{Respuesta:} Se envia la informaci�n critica al ministerio.\\    
    \textbf{Medici�n de la respuesta:} La informaci�n relacionada con la predicci�n se envia en a lo sumo 30 segundos.
    
    \item El procesamiento de los datos obtenidos para generar una predicci�n se realiza lo mas rapido posible.\\
    \textbf{Fuente:} interna (alarma).\\
    \textbf{Estimulo:} Pedido de procesamiento de los datos almacenados.\\
    \textbf{Artefacto:} Estaci�n central (procesamiento).\\
    \textbf{Entorno:} Sobrecarga.\\
    \textbf{Respuesta:} Los datos son procesados y se devuelve un resultado.\\
    \textbf{Medici�n de la respuesta:} El resultado se procesa en menos de 1 minuto.
    \end{itemize}
    
\item Atributos de disponibilidad:
\end{enumerate}