\chapter{Arquitectura}

En esta parte se desarrollar� la arquitectura general del producto. Analiceremos estilos arquitect�nicos, las t�cticas correspondientes para poder contemplar los atributos de calidad y finalmente diferentes estructuras que nos conllevar�n a relizar las vistas arquitect�nicas apropiadas para lograr un buen entendimiento.

\section{Componentes y conectores}

La primera de las estructuras que vamos a analizar es la asociada a entidades run-time. Para ellos utilizaremos vistas de componentes y conectores que nos permitan identificar �stas entidades y poder instanciarlas como componentes o conectores. Esto nos dar� una idea de como los diferentes componentes contribuyen a la ejecuci�n de las funcionalidades solicitadas y como las t�cticas aplicadas se corresponden al control de los diferentes atributos de calidad. Presentaremos tres vistas de acuerdo a los componentes que funcionan en cada uno de las tres grandes partes que componen nuestro sistema(Terminal Remota, Estaci�n central, Sistema de Monitoreo) y tambi�n explicaremos la forma de comunicaci�n y flujo de datos entre dichas partes. Cada una de las tres vistas se encuentra al final de �sta secci�n por lo que es aconsejable analizarla en paralelo a medida que se lee la explicaci�n del funcionamiento de las componentes internas.

\subsection{Funcionamiento de la Estaci�n Central}


\subsubsection{T�cticas elegidas en la Estaci�n Central}


\subsection{Funcionamiento del Sistema de Monitoreo}


\subsubsection{T�cticas elegidas en el Sistema de Monitoreo}


\section{Comunicaci�n entre la Estaci�n Central y el Sistema de Monitoreo}