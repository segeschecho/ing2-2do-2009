\chapter{Informe de avance}


En esta parte de la documentaci�n presentaremos una visi�n global del avance del proyecto, complicaciones, soluciones y otras cuestiones relacionadas con el desarrollo del proyecto.

Lo m�s relevante es el cumplimiento en los tiempos estimados, la preentrega se puedo realizar a tiempo y nos fue de mucha utilidad las correciones para continuar con las etapas siguientes.

Al tiempo en que determin�bamos atributos de calidad, surgi� el QAW de la c�tedra que nos permiti� terminar de cerrar aquellos atributos de calidad m�s relevantes para as� poder comenzar a pensar en t�cticas y finalmente establecer una arquitectura que soporte las necesidades del proyecto.

La arquitectura fue uno de los momentos de mayor debate en el grupo, hubo que investigar y leer todo el material provisto por los docentes para poder seleccionar t�cticas que nos permitiesen solucionar y controlar respuestas a los est�mulos generados por los escenarios de calidad previamente desarrollados. Si embargo la tarea de depuraci�n de la arquitectura fue una de las m�s discutidas ya que implementamos una forma no muy �ptima de llevarla a cabo y que se contrarestaba con la idea de Brooks de centralizar el dise�o de la arquitectura en lo posible en una persona o en un grupo con un marcado l�der. En principio cada integrante del grupo pens� por separado y posteriormente a hablar de lo necesario a dise�ar, una arquitectura propia. Luego se intent� ensamblar y hacer coincidir cada una de �stas arquitecturas lo que fue una tarea complicada. Finalmente optamos por tomar una de ellas(la de mayor completitud) e ir arm�ndola y modific�ndola de acuerdo a las ideas comunes entre todos los integrantes. As� al fin, se logr� establacer una aquitectura consensuada.

La existencia de un parcial en medio del desarrollo del trabajo pr�ctico fue de cierto modo otro problema con el cual lidiar. Si bien no se dej� en ning�n momento el trabajo pr�ctico de lado hubo que distribuir mejor las tareas y tiempos para satisfacer ambos compromisos.

La implementaci�n de la prueba de concepto no nos result� demasiado complicada. Nos pusimos de acuerdo en el lenguaje a utilizar y las tareas pudieron desarrollarse como se hab�an previsto.

Es importante destacar que se llevaron a cabo mitigaciones a algunos de los riesgos m�s relevantes como las problem�ticas en las estimac�ones o en el armado de la arquitectura.

La semana adicional brindada por la c�tedra para la entrega del trabajo pr�ctico nos sirvi� en gran medida para releer la documentaci�n del mismo y realizar las correcciones necesarias en los documentos a entregar, as� como dise�ar test m�s completos para la prueba de concepto.