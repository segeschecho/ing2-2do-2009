\chapter{Requerimientos funcionales y atributos de calidad}

\section{Prop�sito del sistema}

La idea del trabajo pr�ctico consiste b�sicamente en poder desarrollar un software consistente, que pueda satisfacer no s�lo las necesidades b�sicas de nuestro cliente plasmadas en requerimientos funcionales, sino que tambi�n pueda garantizar el cumplimiento de atributos de calidad que constituyen un factor fundamental en el correcto desempe�o de nuestro sistema al momento de salir a producci�n. 

El objetivo de la realizaci�n de dicho software es en principio, poder predecir, mediante la existencia de terminales remotas con sensores de medici�n y modelos matem�ticos de procesamiento, el estado del tiempo en la totalidad del pa�s, poniendo principal �nfasis en generar alertas en caso de tormentas el�ctricas y huracanes a tiempo para poder evacuar las zonas afectadas por dichos fen�menos. Sin embargo es muy fuerte la correspondencia entre las funcionalidades a desarrollar para satisfacer dicho objetivo y atributos de calidad cr�ticos en este contexto. Por ejemplo la disponibilidad es un factor fundamental para el cumplimiento del prop�sito.

En esta parte del informe intentaremos destacar, aquellas funcionalidades necesarias en el sistema, as� como los atributos de calidad instanciados en escenarios que se pudieron obtener del relevamiento del contexto.


\newpage
\section{Generaci�n y descripci'on de Casos de Uso}

En el siguiente texto se mostrar'an los casos de uso principales identificados en el sistema, junto con su descripci'on. En algunos casos, se incluir'an tambi�n los diferentes actores, que participan en el mismo, 'estos se corresponden con los actores que se muestran en el diagrama de contexto del sistema.

\subsection{�ndice de casos de Uso}

\subsubsection{Terminal Remota - TR}

\begin{itemize}
	\item TR01. Configurando agenda en sensor
	\item TR02. Obteniendo datos censados
	\item TR03. Monitoreando/configurando consumo de energ�a 
	\item TR04. Detectando ca�da del sensor
	\item TR05. Detectando recuperaci�n del sensor 
	\item TR06. ABM de sensor
	\item TR07. Detectando posible ca�da por falta de energ�a
	\item TR08. Recuperando luego de ca�da por falta de energ�a
	\item TR09. Sincronizando datos de censados
	\item TR10. Encriptando, comprimiendo y enviando mensaje seg�n protocolo
	\item TR11. Repitiendo mensaje al no tener confirmaci�n
	\item TR12. Borrando mensaje al llegar confirmaci�n
\end{itemize}

\subsubsection{Estaci�n Central - EC}

\begin{itemize}
	\item EC01. Marcando punto de restauraci�n de datos correctos del sistema
	\item EC02. Procesando modelos matem�ticos
	\item EC03. Notificando resultado del modelo
	\item EC04. Guardando datos del modelo
	\item EC05. Triangulando o reemplazando por BS TR/sensor
	\item EC06. Detectando alta de TR/Sensor
	\item EC07. Anunciando remplazo o triangulaci�n de TR/sensor
	\item EC08.	Normalizando TR/sensor y anunci�ndolo
	\item EC09. Detectando red congestionada y anunci�ndolo
	\item EC10. Detectando outliers y anunci�ndolo
	\item EC11.	Descartando � incluyendo posibles outliers
	\item EC12.	Ordenando y descartando mensajes repetidos
	\item EC13.	Confirmando llegada de mensaje
	\item EC14.	Guardando informaci�n de uso de Biggest Satelite
	\item EC15.	Definiendo intervalos de confianza para outliers
	\item EC16.	Obteniendo informaci�n e�lica del sistema del SM
	\item EC17.	Actualizando pautas en los modelos matem�ticos
	\item EC18.	Recibiendo, descomprimiendo y desencriptando mensaje de TR
	\item EC19. Recibiendo informaci�n de Biggest Satelite
	\item EC20.	Actualizando modelos para clientes externos
\end{itemize}

\subsubsection{Sistema de monitoreo - SM}

\begin{itemize}
	\item SM01.	Pidiendo datos del uso del servicio de Biggest Satelite
	\item SM02. Notificando ca�da de TR
	\item SM03.	Notificando red saturada 
	\item SM04.	Enviando datos al Min. de Infraestructura 
	\item SM05.	Enviando datos a cliente externo 
	\item SM06.	Recibiendo informaci�n del sistema e�lico
	\item SM07.	Configurando agenda dentro del sensor de una TR 
	\item SM08.	Configurando env�o de alertas
	\item SM09.	ABM de cliente externo
	\item SM10.	ABM de TR 
	\item SM11.	ABM de sensor 
	\item SM12.	Actualizando pantallas del monitoreo en tiempo real
\end{itemize}

\subsection{Casos de uso dentro de las TRs:}

\begin{enumerate}
\item TR01:
	\begin{itemize}
	\item Agentes: Usuario u Operador, Sensor.
	\item Descripci'on: El Usuario u Operador indica que se tiene que modificar la agenda de uno de los sensores de la TR, esta recibe las instrucciones, y modifica dicha agenda utilizando el protocolo de conexi�n con el sensor.
	\end{itemize}
	
\item TR02:
	\begin{itemize}
	\item Agentes: Sensor.
	\item Descripci'on: La TR recibe un dato de un sensor y lo almacena para luego cuando llegue el tiempo correcto lo env�e a la estaci�n central.
	\end{itemize}
	
\item TR03:
	\begin{itemize}
	\item Agentes: Componente muy usada del mercado, Bater�a, Panel solar)
	\item Descripci'on: La TR recibe informaci�n peri�dica del estado de la Bater�a y del Panel Solar y en base a esa informaci�n administra el consumo de energ�a.
	\end{itemize}

\item TR04:
	\begin{itemize}
	\item Agentes: Estaci'on Central, Sensor, clock interno.
	\item Descripci'on: Se detecta la ca�da de un sensor cuando la TR no recibe datos del mismo dentro del tiempo que tiene configurado para recibirlos. Se notifica esto a la EC para que tome las medidas que considere apropiadas.
	\end{itemize}
	
\item TR05:
	\begin{itemize}
	\item Agentes: Estaci'on Central, Sensor.
	\item Descripci'on: Se reciben datos de un sensor que no estaba emitiendo (estaba roto). Se lo considera reactivado y se empiezan a almacenar normalmente los datos que este produce. Se notifica a la EC.
	\end{itemize}
	
\item TR06:
	\begin{itemize}
	\item Agentes: Sensor, Operario, Estaci'on Central
	\item Descripci'on: Un operario instala un nuevo sensor en la TR y lo configura para enviar datos a la TR. Se notifica a la EC de la presencia de este nuevo sensor. En caso de modificaci�n o baja sucede algo an�logo: el operario hace las modificaciones f�sicas y de configuraci�n pertinentes en la TR y esta comunica el cambio a la EC.
	\end{itemize}

\item TR07:
	\begin{itemize}
	\item Agentes: Bater�a y Panel Solar (Componente muy usada del mercado), Estaci'on Central.
	\item Descripci'on: La TR recibe una alerta de bater�a en nivel m�nimo del sensor de bater�a y una alerta de radiaci�n solar muy baja (no utilizable) del panel solar. Como no puede utilizar ninguna de las dos fuentes de energ�a que posee, notifica a la EC que va a quedarse sin energ�a.
	\end{itemize}

\item TR08:
	\begin{itemize}
	\item Agentes: Bater�a y Panel Solar (Componente muy usada del mercado), Estaci'on Central.
	\item Descripci'on: La TR reanuda sus funciones y comienza a almacenar nuevamente los datos que los sensores le env�an, para luego hacerlos llegar a la estaci�n central.
	\end{itemize}
	
\item TR09:
	\begin{itemize}
	\item Agentes: Clock interno
	\item Descripci'on:  El clock interno de la TR notifica que ha pasado un minuto desde el �ltimo env�o de datos a la Estaci�n Central. La TR recolecta los �ltimos datos obtenidos de los sensores para ser enviados.
	\end{itemize}

\item TR10:
	\begin{itemize}
	\item Agentes: Sistema Interno
	\item Descripci'on: Los datos son comprimidos y encriptados para luego poder ser enviados a la EC seg�n indicaci�n de un protocolo pertinente.
	\end{itemize}	

\item TR11:
	\begin{itemize}
	\item Agentes: Clock interno
	\item Descripci'on: Dado que en la red GSM hay riesgo de p�rdidas y retrasos, la TR tiene configurado un tiempo de espera para recibir la confirmaci�n de que un mensaje enviado fue recibido. Al haber transcurrido este tiempo y no haber recibido esta confirmaci�n, se da el mensaje por perdido y se lo env�a nuevamente.
	\end{itemize}
	
\item TR12:
	\begin{itemize}
	\item Agentes: EC
	\item Descripci'on: Se recibe la confirmaci�n de recepci�n correcta de datos por parte de la EC. Se borran estos datos de la TR (porque ya est�n en la EC).
	\end{itemize}
	
\end{enumerate}

\subsection{Casos de uso dentro de la EC:}

\begin{enumerate}

\item EC1:
	\begin{itemize}
	\item Agentes: Sistema Interno
	\item Descripci'on: En esta funcionalidad la estaci�n central determina un punto de restauraci�n, donde todos los datos hasta dicho punto son correctos.
	\end{itemize}
	

\item EC2:
	\begin{itemize}
	\item Agentes: Sistema Interno
	\item Descripci'on: Con un punto de restauraci�n bien definido y con todos los datos necesarios el sistema procesa los modelos pertinentes.
	\end{itemize}
	

\item EC3:
	\begin{itemize}
	\item Agentes: Sistema Interno
	\item Descripci'on: Una vez que se terminaron de calcular los modelos, y se guardan los resultados, se les notifica  a los interesados.
	\end{itemize}
	

\item EC4:
	\begin{itemize}
	\item Agentes: Sistema Interno
	\item Descripci'on: Una vez finalizados los c�lculos, y obtenidos los resultados de los modelos, la estaci�n central los guarda en la base de datos.
	\end{itemize}

  \item EC05:
	\begin{itemize}
	\item Agentes: Clock interno
	\item Descripci'on: Cuando se detecta la ca�da de una TR o de un sensor de una TR, la central trata de subsanar el problema. Si la ca�da fue de una TR, entonces trata de triangularla utilizando la informaci�n de las TRs vecinas. De no ser posible, se utiliza el servicio de Biggest Satelite para obtener los datos de la zona donde se encuentra la TR ca�da, guardando el tiempo de uso del sistema. Si la ca�da fue de un sensor dentro de una TR, entonces, se trata de triangular el sensor, con la informaci�n de los sensores del mismo tipo, de las TRs vecinas. 
	\end{itemize}
	

\item EC6:
	\begin{itemize}
	\item Agentes: TR  / Biggest Satellite
	\item Descripci'on: La estaci�n central detecta la ca�da de una TR cuando pasado el minuto del ultimo dato recibido, no se reciben mas datos. As� tambi�n de la misma forma, cuando se detecta la llegada de datos de una TR, la central detecta que la misma se reactiv�. Desactiva una triangulaci�n de esa TR si la hubiere o suspende los servicios de Biggest Sattelite para reemplazar esa TR si estuvieran activos. Se toman normalmente los datos provenientes de esa TR.
	\end{itemize}
	
\item EC07:
	\begin{itemize}
	\item Agentes: Sistema de Monitoreo
	\item Descripci'on: Una vez que se realiz� la triangulaci�n o se comenz� a utilizar el servicio de Biggest Satelite, se anuncia al sistema de monitoreo.
	\end{itemize}
	

\item EC08:
	\begin{itemize}
	\item Agentes: Sistema de monitoreo
	\item Descripci'on: Cuando se detecte la reactivaci�n de un sensor, se anuncia al sistema de monitoreo.
	\end{itemize}
	

\item EC09:
	\begin{itemize}
	\item Agentes: Sistema de monitoreo
	\item Descripci'on: Se detecta congesti�n en la red, cuando los datos arriban con retraso a la estaci�n central, en este caso se informa al sistema de monitoreo para que este avise a quien corresponda.
	\end{itemize}
	

\item EC10:
	\begin{itemize}
	\item Agentes: Sistema interno, operario
	\item Descripci'on: Dependiendo de la configuraci�n establecida para detectar outliers en los datos que se reciben, el sistema central, al detectar uno, se encarga de notificar dicho suceso a los responsables, para que lo examinen, y determinen si es un dato correcto o no.
	\end{itemize}
	

\item EC11:
	\begin{itemize}
	\item Agentes: Experto en meteorolog�a
	\item Descripci'on: Luego de la notificaci�n de outliers, el encargado correspondiente de analizarlos, le indica al sistema que se descarten o se incluyan para procesar los modelos.
	\end{itemize}
	

\item EC12:
	\begin{itemize}
	\item Agentes: TR
	\item Descripci'on: Ante el arribo de mensajes repetidos o en desorden, la EC se encarga de ordenarlos y descartar los repetidos.
	\end{itemize}


\item EC13:
	\begin{itemize}
	\item Agentes: TR
	\item Descripci'on: Al recibir un mensaje de una TR, se le env�a un mensaje de confirmaci�n para que esta sepa que el mensaje llego correctamente.
	\end{itemize}
	
\item EC14:
	\begin{itemize}
	\item Agentes: Sistema interno
	\item Descripci'on: Cada vez que se termina de utilizar el servicio de Biggest Satelite, la estaci�n central, guarda el tiempo utilizado, para que luego se realice la liquidaci�n.
	\end{itemize}
	

\item EC15:
	\begin{itemize}
	\item Agentes: Experto en modelos mate maticos, Sistema de monitoreo
	\item Descripci'on: El experto configura los intervalos de confianza para outliers que �l considera adecuados. El sistema guarda esta configuraci�n.
	\end{itemize}
	

\item EC16:
	\begin{itemize}
	\item Agentes: Sistema de monitoreo.
	\item Descripci'on: El sistema de monitoreo le env�a a la EC los datos provenientes del sistema e�lico para que esta los pueda usar seg�n corresponda.
	\end{itemize}
	

\item EC17:
	\begin{itemize}
	\item Agentes: Experto en modelos matem�ticos, Sistema de monitoreo
	\item Descripci'on: Se ingresan o se eliminan reglas al modelo matem�tico, agregando de esta forma la posibilidad de realizar todo tipo de c�lculos.
	\end{itemize}
	
\item EC18:
	\begin{itemize}
	\item Agentes: Sistema interno
	\item Descripci'on: Se reciben los mensajes de las distintas TRs y pasan por las capas de transporte correspondientes para descomprimir y desencriptar.
	\end{itemize}


\item EC19:
	\begin{itemize}
	\item Agentes: Biggest Satelite
	\item Descripci'on: El sistema BS le env�a a la EC los datos pedidos luego del reemplazo de una o varias TRS o sensores.
	\end{itemize}
	

\item EC20:
	\begin{itemize}
	\item Agentes: Cliente Externo, experto en modelos matem�ticos
	\item Descripci'on: Mediante un mecanismo fijado se agregan nuevos modelos matem�ticos para ser procesados luego para un cliente externo.
	\end{itemize}
	
\end{enumerate}

\subsection{Casos de uso dentro del sistema de Monitoreo}

\begin{enumerate}
\item SM01:
	\begin{itemize}
	\item Agentes: Clock interno (tarea programada)
	\item Descripci'on: El �ltimo d�a h�bil del mes el sistema env�a al sistema de facturaci�n, la cantidad de tiempo de uso del servicio de Biggest Satelite.
	\end{itemize}

\item SM02:
	\begin{itemize}
	\item Agentes: Ec, Sistema de mail
	\item Descripci'on: Cuando la estaci�n central detecta la ca�da de una TR, el sistema de monitoreo es notificado para que env�e un mail a los responsables avisando que una TR no esta respondiendo.
	\end{itemize}

\item SM03:
	\begin{itemize}
	\item Agentes: EC, Sistema de mail
	\item Descripci'on: Cuando la estaci�n central detecta que la red se encuentra saturada, esta le notifica al sistema de monitoreo, para que env�e un mail a los responsables avisando de la eventualidad.
	\end{itemize}

\item SM04:
	\begin{itemize}
	\item Agentes: Ministerio de Infraestructura, Base de datos
	\item Descripci'on: Mediante el servicio de Web Service, se le env�an los datos que el sistema del Ministerio de Infraestructura pide para mostrar en su p�gina.
	\end{itemize}

\item SM05:
	\begin{itemize}
	\item Agentes: Cliente Externo, Base de Datos
	\item Descripci'on: Mediante el sistema de Web Service, se env�an los datos solicitados al cliente externo, para que los utilice.
	\end{itemize}

\item SM06:
	\begin{itemize}
	\item Agentes: Sistema e�lico, Base de datos, EC
	\item Descripci'on: Se recibe la informaci�n del sistema e�lico, y se almacena en la base de datos, para que luego la estaci�n central los procese. Se notifica a la EC para que luego esta compare los datos con los obtenidos de las TRs.
	\end{itemize}

\item SM07:
	\begin{itemize}
	\item Agentes: Base de Datos, EC, TR, operario
	\item Descripci'on: Desde el sistema de monitoreo y a trav�s de la Estaci�n Central, un operario puede modificar la agenda de un sensor que elija dentro de una TR. Cuando se realiza el cambio, esto es reflejado en el Servidor de Base de datos, para que quede registrado el cambio.
	\end{itemize}

\item SM08:
	\begin{itemize}
	\item Agentes: Operario, Base de Datos
	\item Descripci'on: Un operario puede configurar el env�o de alertas para cuando la Estaci�n Central detecta la ca�da de una TR o la saturaci�n de la red, agregando o quitando responsables en la lista de mails de notificaci�n. Este cambio se refleja en el Servidor de Base de datos.
	\end{itemize}

\item SM09:
	\begin{itemize}
	\item Agentes: Operario, Base de Datos
	\item Descripci'on: Un operario puede ingresar o eliminar a un cliente externo, el cual podra recibir o no la informaci�n que el sistema procese. Esto se refleja en el Servidor de Base de datos.
	\end{itemize}

\item SM10:
	\begin{itemize}
	\item Agentes: Base de Datos, operario, EC
	\item Descripci'on: Un operario puede ingresar o quitar una TR, con toda su informaci�n relacionada (ubicaci�n, identificaci�n, etc.).Este cambio se refleja en el Servidor de Base de datos y en la Estaci�n Central, la cual comenzar� o no a recibir y procesar los datos que lleguen o no de dicha TR.
	\end{itemize}

\item SM11:
	\begin{itemize}
	\item Agentes: Base de Datos, operario, EC
	\item Descripci'on: Un operario puede agregar o quitar un nuevo tipo de sensor. Dicho tipo es reflejado en el Servidor de Base de datos y en la Estaci�n Central.
	\end{itemize}

\item SM12: 
	\begin{itemize}
	\item Agentes: Operario, Base de Datos, Sistema de Pantallas
	\item Descripci'on: Se muestra en pantallas seg�n sea pedido las distintas caracter�sticas que el usuario desea ver.
	    \begin{itemize}
	      \item Vista global del estado sistema
	      \item Vista por segmento de red
	      \item Vista por TR o sensor
	    \end{itemize}
	\end{itemize}
	
\end{enumerate}
\clearpage

\newpage
\chapter{Escenarios}
\clearpage