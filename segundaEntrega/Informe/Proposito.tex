\chapter{Objetivo del sprint}

La caracter�stica principal de �ste sprint es la aparici�n de nuevos requerimientos funcionales.  Esto hace que se tenga que pensar como se adec�a la arquitectura a dichos requerimientos y en caso de ser necesario, c�mo deber�a modificarse para contemplar los mismos, teniendo en cuenta que funcione todo lo hecho hasta el momento. Entonces, b�sicamente el objetivo del sprint es adaptar la arquitectura pensada en la primera entrega del trabajo pr�ctico a los nuevos requerimientos, y llevar a cabo la nueva planificaci�n utilizando scrum. Una vez hecho esto debemos seleccionar la lista de stories que llevaremos a cabo en el sprint, estimarlos (utilizando valores relativos, que luego se traducir�n a esfuerzo) y comenzar con su implementaci�n para lograr un adecuado product increment que sea aceptado por el usuario.

Ser� necesario pensar c�mo van a impactar los nuevos requerimientos en la arquitectura ya dise�ada y como realizar un adecuado trade off para seguir garantizando el cumplimiento de los antiguos requerimientos y la satisfacci�n de los nuevos.

Tambi�n se espera tener los primeros stories realizados, que se corresponden con los que se indican en el enunciado del trabajo, resolviendo los problemas que surjan en el trayecto del mismo. Este avance se ver� reflejado en el Sprint Burndown Chart final, y se podr� observar como fue evolucionando el sprint durante las 2 semanas que dura.

Se espera fijar ademas las ideas cuando se tengan dudas acerca del proyecto su avance y su desarrollo cuando tengamos las stand-up meetings correspondientes, corroborando el seguimiento del proyecto.