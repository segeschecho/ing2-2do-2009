\chapter{Planificaci�n}	


\section{Epics}

Los Epics identificados estan relacionados con los principales cambios en los requerimientos que tuvo el sistema para esta etapa. Los Epics, son similares a casos de uso, pero de muy alto nivel. Estos nos ayudar�n a agrupar tareas, que luego asignaremos a cada uno de los User Story, y poder armar los sprints, en particular el primero, que entra para esta nueva entrega del trabajo.

\begin{itemize}
\item E01 - Configurando conjunto de TRs Regionales.
\item E02 - Configurando colaboraci�n entre regiones.
\item E03 - Configurando procesamiento de modelos en forma distribuida.
\item E04 - Evaluando modelos con distintos algoritmos.
\item E05 - Suscribiendo modelos a Trs
\end{itemize}


\section{Product Backlog}

En esta secci�n se detallar�n los User Stories relacionados con los nuevos requerimientos. A su vez cada una, est� incluida dentro de alguno de los Epics descriptos anteriormente, con el objetivo de facilitar la compresion de las diferentes tareas.

\begin{itemize}

\item{US01 - E01:}\\
\begin{tabular}{|l p{12cm}|}
\hline
\textbf{Como:} & Empresa regional.\\
\textbf{Quiero:} & Estar a cargo de la parametrizaci�n ejecuci�n y resultados de mis modelos.\\
\textbf{Para:} & Trabajar de acuerdo a nuestros criterios.\\
\hline
\end{tabular}


\item{US02 - E01:}\\
\begin{tabular}{|l p{12cm}|}
\hline
\textbf{Como:} & Empresa regional.\\
\textbf{Quiero:} & Poder incorporar nuevas TR a mi zona.\\
\textbf{Para:} & Obtener los datos necesarios para mis modelos.\\
\hline
\end{tabular}


\item{US03 - E02}\\
\begin{tabular}{|l p{12cm}|}
\hline
\textbf{Como:} & Empresa regional\\
\textbf{Quiero:} & Trabajar en colaboraci�n con otras regiones.\\
\textbf{Para:} & Poder solicitar datos  o resultados parciales procesados por los modelos de las mismas.\\
\hline
\end{tabular}


\item{US04 - E03}\\
\begin{tabular}{|l p{12cm}|}
\hline
\textbf{Como:} & Usuario del sistema.\\
\textbf{Quiero:} & Incrementar el poder de c�mputo.\\
\textbf{Para:} & Que aquellos modelos que as� lo requieran puedan llevar a cabo su procesamiento de manera �ptima.\\
\hline
\end{tabular}


\item{US05 - E02}\\
\begin{tabular}{|l p{12cm}|}
\hline
\textbf{Como:} & Usuario del sistema.\\
\textbf{Quiero:} & Atacar el incremento de congesti�n asociada a los nuevos requerimientos.\\
\textbf{Para:} & El sistema siga funcionando correctamente a tiempo sin saturar la red GSM.\\
\hline
\end{tabular}


\item{US06 - E01}\\
\begin{tabular}{|l p{12cm}|}
\hline
\textbf{Como:} & Usuario del sistema.\\
\textbf{Quiero:} & Que las terminales remotas funcionen con s�lo un subconjunto de los sensores disponibles en los casos en que as� fuese necesario.\\
\textbf{Para:} & No utilizar informaci�n innecesaria.\\
\hline
\end{tabular}


\item{US07 - E03}\\
\begin{tabular}{|l p{12cm}|}
\hline
\textbf{Como:} & Intendente.\\
\textbf{Quiero:} & Quiero que se utilicen los equipos disponibles.\\
\textbf{Para:} & Mejorar el poder de c�mputo para los modelos que as� lo requieran.\\
\hline
\end{tabular}

   
\item{US08 - E03}\\
\begin{tabular}{|l p{12cm}|}
\hline
\textbf{Como:} & Usuario del sistema.\\
\textbf{Quiero:} & Que se particionen el conjunto de reglas.\\
\textbf{Para:} & Procesarlas concurrentemente en los equipos disponibles.\\
\hline
\end{tabular}

 
\item{US09 - E03}\\
\begin{tabular}{|l p{12cm}|}
\hline
\textbf{Como:} & Usuario del Sistema.\\
\textbf{Quiero:} & Que se pueda configurar de manera sencilla la colaboraci�n entre subpartes.\\
\textbf{Para:} & Para lograr un f�cil intercambio.\\
\hline
\end{tabular}
  

\item{US010 - E03}\\
\begin{tabular}{|l p{12cm}|}
\hline
\textbf{Como:} & Usuario del Sistema.\\
\textbf{Quiero:} & Que se pueda configurar de manera sencilla la colaboraci�n entre modelos.\\
\textbf{Para:} & Para lograr un f�cil intercambio.\\
\hline
\end{tabular}


\item{US011 - E04}\\
\begin{tabular}{|l p{12cm}|}
\hline
\textbf{Como:} & Usuario del Sistema.\\
\textbf{Quiero:} & Poder tener varios algoritmos de evaluaci�n de reglas.\\
\textbf{Para:} & Para chequear consistencia de modelos.\\
\hline
\end{tabular}



\item{US012 - E05}\\
\begin{tabular}{|l p{12cm}|}
\hline
\textbf{Como:} & Usuario del Sistema.\\
\textbf{Quiero:} & Que los modelos puedan suscribirse a las TRs:\\
\textbf{Para:} & Obtener solo los datos necesarios para su procesamiento.\\
\hline
\end{tabular}

\end{itemize}
%\textbf{Como:}
%\textbf{Quiero:}
%\textbf{Para:}
%Como usuario quiero Realizar una especificaci�n clara de las modificaciones arquitect�nicas para adaptar nuestro sistema a los nuevos requerimientos.
%\textbf{Como:}
%\textbf{Quiero:}
%\textbf{Para:}
%Como usuario quiero realizar Product Backlog para tener todos los stories requeridos para realizar todo el proyecto.
%\textbf{Como:}
%\textbf{Quiero:}
%\textbf{Para:}
%Como usuario quiero realizar el Sprint Backlog  con las stories seleccionadas, su estimaci�n, criterios de aceptaci�n y descomposici�n en tareas para tener las cosas que hay que hacer para la iteraci�n actual.
%\textbf{Como:}
%\textbf{Quiero:}
%\textbf{Para:}
%Como usuario quiero realizar una documentaci�n del seguimiento de proyecto  usando burndown charts para tener una visi�n grafica del avance de la iteraci�n.
%\textbf{Como:}
%\textbf{Quiero:}
%\textbf{Para:}
%Como usuario quiero queremos realizar un dise�o de objetos para resolver el problema de la ejecuci�n de los modelos matem�ticos y puedan ser ejecutados en forma distribuida.
%\textbf{Como:}
%\textbf{Quiero:}
%\textbf{Para:}
%Como usuario quiero proyecto queremos realizar una comparaci�n del trabajo realizado con UP y SCRUM para ver las ventajas y desventajas de cada uno de las metodolog�as.
%\textbf{Como:}
%\textbf{Quiero:}
%\textbf{Para:}


