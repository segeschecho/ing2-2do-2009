\documentclass{beamer}

%\usetheme{Boadilla}
\usepackage[spanish,activeacute]{babel}
\usepackage[latin1]{inputenc}

\title[No silver bullet]{No Silver Bullet. Essence and Accidents in Software Engineering.}
\author[Castillo, Elizalde, Gonzalez, Page]{Gonzalo Castillo, Victoria Elizalde, Segio Gonzalez, Mart��n Page}
\date[10 de Noviembre 2009 ]{Fred Brooks es Cient��fico de la Computaci�n e Ingeniero de Sofware. Recibi� el Turing Award en 1999.}
\institute{FCEyN}

\begin{document}

%\frame{\titlepage}

\section{}
\subsection{}
\frame
{
  \frametitle{Introducci�n}
  \begin{itemize}
  \item [El paper] No Silver Bullet. Essence and Accidents in Software Engineering. IEEE Computer, Abril de 1987.
  \item [Somos] Gonzalo Castillo, Victoria Elizalde, Sergio Gonzalez y Mart�n Page
  \item [Bio] Fred Brooks es Cient��fico de la Computaci�n e Ingeniero de Sofware. Recibi� el Turing Award en 1999 y es conocido por haber escrito el libro The Mythical Man-Month, adem�s de No silver bullet.
  \end{itemize}
}
\frame
{
  \frametitle{La met�fora}
  \begin{itemize}
  \item <1->Brooks compara un proyecto de software con un hombre lobo: algo inocente se transforma en un monstruo
  \item <2-> Necesidad de una "bala de plata", algo que haga bajar costos y aumente productividad, confiabilidad y simplicidad
  \item <3-> Mayor dificultad del sofware: la especificaci�n, dise�o y testing de la parte conceptual
  \end{itemize}
}
\frame
{
  \frametitle{Las dificultades}
  \begin{itemize}
  \item<1-> Dificultades Esenciales: Inherentes a la naturaleza misma del software
  	\begin{itemize}
  \item<2-> Complejidad
  \item<3-> Conformidad
  \item<4-> Modificabilidad    
  \item<5-> Invisibilidad
  \end{itemize}
  \item<6-> Dificultades accidentales: Dificultades no inherentes al sotware sino a su producci�n
  \end{itemize}
}

\frame
{
  \frametitle{Avances que resolvieron dificultades accidentales}
  \begin{itemize}
  \item<1-> Lenguajes de Alto Nivel
  \item<2-> Time-Sharing
  \item<3-> Ambientes de desarrollo unificado
  \end{itemize}
}
\frame
{
  \frametitle{Conclusiones}
  \begin{itemize}
  \item<1-> Bla bla
  \item<2-> Saraza
  \item<3-> Mas guitarra     
  \end{itemize}
}
\end{document}
